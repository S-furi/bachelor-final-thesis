%!TEX root = ../thesis-main.tex
\chapter{Implementazione e Verifica}\label{chap:implementation}

\section{Superamento delle specifiche GraphQL}\label{sec:overcoming-gql-specs}
Come illustrato nella sezione \Cref{ssec:server-gql-apis}, l'approccio desiderabile di sviluppo del codice per lo schema GraphQL è un approccio di
tipo \textit{code-first}. La libreria GraphQL Kotlin \footnote{https://opensource.expediagroup.com/graphql-kotlin/} offre proprio questo tipo di
approccio, permettendo di astrarre il più possibile dalla implementazione dello schema. Questo però comporta inevitabilmente un conflitto tra il modello
di Alchemist e le specifiche GraphQL, le quali non sono in grado di fornire una rappresentazione uno a uno del modello.
In questo contesto nasce la necessità di definire dei surrogati per adattare il modello di Alchemist ad uno schema che sia conforme alle specifiche,
e allo stesso tempo garantire la genericità dipendente dalle diverse \textit{incarnation}.

\subsection{Surrogati (\textit{Object Adapter Design Pattern})}\label{ssec:surrogates}
I surrogati per il modello di Alchemist hanno il compito di adattare elementi complessi del dominio con versioni meno complesse e compatibili
con uno schema GraphQL. Nel contesto dell'architettura client-server, oltre a ricoprire questo ruolo,
i surrogati hanno anche la funzione di \ac{DTO}, poiché solo essi compariranno all'interno delle comunicazioni in rete.

Siccome la generazione dello schema GraphQL è delegata alla libreria impiegata, è importante notare come i tipi di Kotlin siano interpretati e adattati
allo schema GraphQL. Di seguito sono riportate le trasformazioni tra tipi di Kotlin e tipi GraphQL:
\begin{itemize}
    \item \textbf{Tipi Primitivi}: i tipi primitivi di Kotlin come \texttt{Int}, \texttt{Float} e \texttt{Double}, \texttt{String} e \texttt{Boolean}
        sono nativamente supportati da GraphQL, permettendo quindi un adattamento immediato tra codice Kotlin e schema GraphQL.
    \item \textbf{Classi}: tutte le classi definite in Kotlin, vengono interpretate all'interno dello schema come di tipo \texttt{type}. Un \texttt{type}
        può comparire come valore di ritorno di un campo dello schema GraphQL, e può estendere un'interfaccia.
    \item \textbf{Metodi e Proprietà}: i metodi e proprietà vengono trasposti in uno schema GraphQL come campi di una struttura di tipo
        \texttt{type}, solo se essi risultano pubblici e non annotati con \kotlin{@GraphQLIgnore}.
    \item \textbf{Input}: ogni classe che figura come parametro all'interno di un metodo (se pubblico e non annotato con \kotlin{@GraphQLIgnore}) viene
        interpretato all'interno dello schema GraphQL come di tipo \texttt{input}. Un tipo \texttt{input} può contenere a sua volta dei campi,
        ma non può estendere un'interfaccia.
    \item \textbf{Interfacce}: tutte le interfacce definite tali all'interno del codice Kotlin, vengono interpretate come tipo \texttt{interface} all'interno
        dello schema GraphQL. Un'interfaccia si compone di campi all'interno dello schema, può figurare come valore di ritorno di un metodo o di una proprietà
        ma \textit{non} può essere un parametro di un metodo.
\end{itemize}
Durante la fase di generazione dello schema a partire dal codice Kotlin quindi, viene stabilita la corretta relazione tra tipo Kotlin e tipo GraphQL, e
in caso in caso di incompatibilità la generazione dello schema fallisce. Se considerassimo per esempio il metodo \kotlin{Node<T>.getConcentration(m: Molecule): T},
è da subito evidente che è incompatibile con uno schema GraphQL, sia per l'utilizzo dell'interfaccia \kotlin{Molecule} come parametro, sia
per l'impiego del generico \kotlin{<T>} (il tipo di \kotlin{Concentration}).

Il modello di Alchemist, facendo pesantemente utilizzo del polimorfismo attraverso interfacce e classi astratte, risulta quindi immediatamente
incompatibile con uno schema GraphQL senza l'intervento di un livello ulteriore di astrazione. In questo scenario si interpongono i surrogati delle 
classi di Alchemist, definiti come in \Cref{lst:gql-surrogate}. Ogni surrogato è generico per il tipo \kotlin{<S>}, ovvero l'oggetto Alchemist
per il quale viene costruito l'\textit{adapter}. In questo modo, possiamo definire un modello surrogato come illustrato nel diagramma delle classi in
\Cref{fig:surrogates-uml-full}.

\code{Kotlin}{./listings/GraphQLSurrogate.kt}{Definizione di un Surrogato Generico GraphQL per il modello di Alchemist.}{gql-surrogate}

\paragraph{Surrogati e Comunicazioni da Client a Server}
Costruendo i surrogati in questo modo, è possibile mappare un elemento del modello di Alchemist ad un surrogato, ma non viceversa. Per tutte le operazioni
che prevedono l'invio di un surrogato dal client infatti, non sono presenti alcuni riferimenti all'oggetto originario (poiché si ricorda essere annotato con 
\kotlin{@GraphQLIgnore open val origin: S}), portando il client ad un'impossibilità di creare e spedire un surrogato Alchemist così definito.
A questo punto quindi, per tutte le operazioni che prevedono l'impiego di un surrogato in input (principalmente
\textit{mutation}) è necessario definire una nuova classe istanziabile dal client e presente nello schema GraphQL con il tipo \texttt{input}. 
In questo modo il client è in grado di definire surrogati e utilizzarli come input di operazioni, e riprendendo l'esempio di prima, il metodo
per recuperare una concentrazione data una molecola sarà mappato in termini di surrogati in \kotlin{NodeSurrogate<T>.getConcentration(MoleculeInput): String},
dove \kotlin{MoleculeInput} è l'oggetto in input istanziabile dal client. A questo punto sarà il server a definire le strategie di conversione
tra oggetti di tipo \texttt{input} e il l'oggetto Alchemist corrispondente.

Per evitare la modifica del codice sorgente delle \ac{API} del modello di Alchemist, grazie alle \textit{extension functions} di Kotlin è possibile
definire funzioni del tipo \kotlin{toXxxGraphQLSurrogate()} in grado di aggiungere un nuovo metodo ad un oggetto senza la necessità di ereditarietà
o l'impiego di design patterns. In \Cref{lst:node-surrogate} è mostrata la funzione di conversione tra un nodo Alchemist e il relativo
surrogato GraphQL, la quale non ha argomenti in input e crea un \texttt{NodeSurrogate} fornendo un riferimento a se stesso e l'\texttt{id} stesso del nodo.

\code{Kotlin}{./listings/ToNodeSurrogate.kt}{Conversione da \texttt{Node} al suo relativo surrogato GraphQL}{node-surrogate}

\subsection{Gestione del Polimorfismo}\label{ssec:generics-serialize}
Come definito in \Cref{ssec:model-adaptation}, le principali sfide da affrontare per l'adattamento del modello di Alchemist con uno schema che sia
compatibile con le specifiche GraphQL sono principalmente l'impossibilità di utilizzare interfacce come input, e la mancanza di supporto per i tipi generici.
Per quanto riguarda la prima, è risolvibile mediante i surrogati, mentre per la seconda è necessario un ragionamento più approfondito riguardo le motivazioni
dietro l'uso di tali tipi generici.

\paragraph{Position}
In Alchemist, una posizione è definita come \kotlin{Position<P : Position<P>>}, utilizzando il cosiddetto "\textit{recursive bound}". Questa tecnica è
principalmente utilizzata per imporre una struttura gerarchica specifica sul tipo delle posizioni. Ciò significa che ogni classe che implementa questa
interfaccia definisce un tipo di \texttt{Position} raffinandone la definizione attraverso l'ereditarietà, consentendo una modellazione più dettagliata,
adattata al dominio di uno specifico contesto (ad esempio, \texttt{Position2D} per modellazioni di spazi bidimensionali).

Poiché la gerarchia delle posizioni è piuttosto limitata e ben definibile, per poter convertire una generica \texttt{Position} è sufficiente fornire
allo schema GraphQL un surrogato per una posizione generica, e almeno una sua realizzazione concreta. Nel sistema raffigurato in \Cref{fig:surrogates-uml-full}
è mostrato come si forniscano due realizzazioni dell'interfaccia \texttt{PositionSurrogate}: \texttt{GenericPositionSurrogate} e \texttt{Position2DSurrogate}.
%
\code{Kotlin}{./listings/PositionSurrogateUtils.kt}{Conversione di una generica \texttt{Position} Alchemist, nel suo relativo surrogato.}{pos-utils}
%
In questo modo, data una generica istanza di una \texttt{Position} all'interno di Alchemist, possiamo mappare il corretto surrogato GraphQL come mostrato in \Cref{lst:pos-utils}
a partire dal numero di dimensioni dello spazio della posizione. Questo esempio mira a delineare che, per ogni tipo di posizione, sarà sufficiente
definire un surrogato basilare e una strategia di conversione per adattare un nuovo tipo di posizione (e.g. \texttt{GeoPosition} composto
da latitudine e longitudine).

Il vantaggio principale nell'impiego di un'interfaccia all'interno di GraphQL, oltre alla definizione di un super-tipo per una serie di oggetti,
risiede nella possibilità di utilizzare gli ``\textit{inline fragments}''. Questo operatore, utilizzabile all'interno di operazioni GraphQL, viene impiegato
quando il valore di ritorno di un campo è un tipo composito (i.e. \texttt{interface} e \texttt{union}) e permette di specificare selezioni di campi differenti
in base al tipo tornato a \textit{runtime}. In \Cref{lst:gql-fragments} è mostrato l'utilizzo di \textit{inline fragments} per le diverse possibili implementazioni
di \texttt{Position}.

\code{text}{./listings/Fragments.graphql}{Utilizzo di \textit{inline fragments} per il tipo \texttt{PositionSurrogate}}{gql-fragments}

\paragraph{Concentration}
Il tipo delle concentrazioni in Alchemist, dipende esclusivamente dal tipo di \textit{Incarnation} selezionta, come descritto in \Cref{ssec:alchemist}.
Diversamente da quanto accade per le posizioni, questo tipo non ha una interfaccia comune da cui tutte le classi ereditano, e perciò risulta più complessa
la sua rappresentazione all'interno di uno schema GraphQL.

Per ogni tipo non facilmente definibile all'interno di uno schema GraphQL, è possibile impiegare uno speciale tipo di dato chiamato
\textit{Custom Scalar}\footnote{http://spec.graphql.org/draft/\#sec-Scalars.Custom-Scalars}. Questo dato può non essere conforme alle specifiche, ma è
necessario definirne strategie di interpretazione lato server prima di essere spedito, e allo stesso tempo definire lato ricevente come interpretare
il \textit{custom scalar} ricevuto.

Solitamente per inserire un custom scalar all'interno di un comunicazione GraphQL, esso viene serializzato come oggetto JSON, in modo tale da avere
una rappresentazione univoca lato ricezione. In questo contesto si inserisce la libreria \texttt{kotlinx.serialization} in grado di fornire le funzionalità
di serializzazione e deserializzazione anche in ambienti \textit{multiplatform}. \texttt{kotlinx.serialization} offre implementazioni di default per
i tipi primitivi di Kotlin, comprendendo anche gran parte delle \textit{Collections} del linguaggio, risultando quindi una valida scelta per tutti quei
casi dove il tipo delle concentrazioni è semplice. Per tutti i casi dove invece non è possibile effettuare una serializzazione con il \textit{serializer}
di default, è necessario definirne una esplicita implementazione.

L'idea di fondo è quella di fornire un surrogato nel \textit{source-set} \textit{commonMain} per il tipo \texttt{T} aderente alle specifiche GraphQL, e allo
stesso tempo includere un \textit{serializer} di \texttt{kotlinx.serialization}: questo può essere fatto semplicemente includendo nella
dichiarazione della classe l'annotazione \kotlin{@Serializable}. Definendo così il surrogato del contenuto della concentrazione, si è in grado di
includere, spedire e ricevere correttamente un tipo \textit{custom} di GraphQL, serializzato in formato JSON per motivi di compatibilità con i protocolli
impiegati dalle librerie. Allo stato attuale, se un \textit{serializer} non viene fornito per il tipo \texttt{T}, allora viene spedita la sua rappresentazione
in stringa. In \Cref{lst:conc-serializer} è mostrato come venga effettuata la serializzazione del contenuto di una concentrazione.
%
\code{Kotlin}{./listings/Serialization.kt}{Serializzazione del contenuto di una concentrazione in formato JSON}{conc-serializer}
%
Attraverso la chiamata a funzione \kotlin{serializer(content::class.java)}, si cerca di utilizzare il \texttt{serializer} per la classe del tipo \texttt{T},
e in caso non venga trovato, verrà lanciata una \texttt{SerializationException}, la quale verrà catturata per poter fornire una rappresentazione in stringa
del tipo.

Per motivi di semplicità e leggerezza dello schema e della notazione, siccome il contenuto delle concentrazioni è sempre serializzato in formato JSON sotto forma di stringa,
non è necessario definire un nuovo \textit{custom scalar} per rappresentarlo, ma è sufficiente considerarlo come una semplice stringa, opportunamente dettagliando la
documentazione riguardo che cosa è rappresentato all'interno della stringa stessa.

\section{Funzionamento}\label{sec:functioning}
\subsection{Web Server}\label{ssec:web-server}
Il servizio di \ac{API} è reso disponibile all'esterno attraverso un web server implementato attraverso \textit{Ktor} \footnote{https://ktor.io/}. Questa libreria
offre un \textit{framework} leggero e altamente configurabile per al creazione di servizi web e, più in generale, di applicazioni \textit{backend}. Generalmente un
servizio Ktor si compone di un insieme di moduli, ognuno dei quali solitamente ricopre ruoli specifici per la messa in funzione di un servizio web. Il \textit{backend}
per l'esposizione di \ac{API} GraphQL comprende 2 moduli essenziali: un modulo di \textit{routing} e un modulo per la configurazione di GraphQL.
Il primo modulo ha il compito di esporre i tre \textit{endpoint} principali su cui opera il server GraphQL:
\begin{itemize}
    \item \texttt{/graphql}: \textit{endpoint} dal quale arrivano le richieste GraphQL di tipo \textit{query} e \textit{mutation}. Le richieste devono
        essere effettuate tramite il protocollo HTTP, preferibilmente attraverso il metodo POST includendo nel payload l'operazione richiesta.
    \item \texttt{/sdl}: \textit{endpoint} per l'introspezione del \ac{SDL}, ovvero lo schema GraphQL esposto dal server, contenente il modello e le operazioni fornite.
    \item \texttt{/subscriptions}: \textit{endpoint} speciale per l'esecuzione di operazioni del tipo \textit{subscription}. A differenza degli altri, questo è interrogabile
        solo attraverso il protocollo \textit{WebSocket}, affinché sia possibile mantenere una connessione persistente in modo efficace.
\end{itemize}

È all'interno di questo modulo che le richieste client arrivano e vengono approvate (i.e. verifica se le richieste sono ben formate). Successivamente queste vengono inviate
al modulo responsabile dell'esecuzione delle operazioni GraphQL, il secondo modulo di cui si compone il web server. Il suo compito principale è quello di validare le richieste
al di sopra dello schema, e in caso di esito positivo, provvedere al recupero dei dati richiesti e infine l'invio della risposta al client richiedente. All'interno di questo modulo
risiede il cuore dell'architettura del server, le cui funzioni stanno alla base del successo dell'impiego delle \ac{API} GraphQL, come esposto nella sezione seguente.

\subsection{Esecuzione e Interrogazione delle API}\label{ssec:apis-functioning}
Le sezioni precedenti hanno illustrato il processo di ideazione, design e implementazione. In questa sezione verrà coperto il funzionamento generale dell'intera architettura,
sia lato server, sia l'interrogazione da parte del client e l'interpretazione dei risultati ottenuti.

\paragraph{Server}
Procedendo attraverso un approccio \textit{code-first}, il server prima di essere messo in funzione effettua un'operazione di generazione dello schema, grazie ad una componente
software denominata \texttt{SchemaGenerator} implementata attraverso la libreria GraphQL Kotlin. Il compito di questa componente è di generare a \textit{runtime} tutte le componenti
del modello indicate all'interno del \textit{build} file in modo tale che siano utilizzabili all'interno di un architettura GraphQL. Verrà quindi dapprima effettuato un controllo
delle componenti sulle specifiche GraphQL, e se il controllo va a buon fine, vengono definiti i relativi \textit{resolver} e \textit{fetcher} per ogni componente elaborata.
Il primo di questi ha il compito di interpretare le richieste GraphQL in arrivo che riguardano quel determinato elemento del dominio, effettuando quindi un \textit{parsing}
della richiesta stessa e dopo aver interpretato il suo contenuto ridireziona la richiesta al \textit{fetcher}, il quale invece ha il compito di recuperare i dati
richiesti dall'operazione all'interno del sistema, secondo la struttura definita all'interno della \textit{query} inviatagli dal client. In caso invece la richiesta venga validata
negativamente, un errore verrà spedito, in modo tale che sia possibile un'interpretazione dell'errore anche lato client.

Questo è quanto accade con operazioni del tipo \textit{query} e \textit{mutation}. Per tutte le operazioni di \textit{subscription}, è necessario stabilire l'evento che scaturisce
l'invio di un dato aggiornato. Come precedentemente anticipato in \Cref{ssec:server-arch}, mediante \texttt{EnvironmentSubscriptionMonitor} è possibile ottenere una versione
dell'ambiente aggiornata ad ogni step che la simulazione esegue. Questo viene effettuato mediante una chiamata periodica da parte della simulazione ad ogni \texttt{OutputMonitor}
ad essa collegato, del metodo \kotlin{stepDone}, appunto ogni volta che uno step viene effettuato. In \Cref{fig:monitor-seq} è illustrato un diagramma di sequenza dettagliato
per ogni operazione di \textit{subscription} effettuata dal client.

\centerimage{./figures/Monitor-Sequence.png}{Diagramma di sequenza del funzionamento di una \textit{subscription} mediante l'uso di \texttt{EnvironmentSubscriptionMonitor}}{monitor-seq}{1.0}

Prima della messa in funzione del server, l'istanza di \kotlin{EnvironmentSubscriptionMonitor} viene collegata alla simulazione e preserva un riferimento all'interno del server GraphQL.
Una volta che la simulazione viene azionata, è possibile effettuare \textit{subscription}. Queste ultime prevedono l'uso di strutture in grado di emettere una serie di valori in modo 
asincrono e duraturo nel tempo: i \texttt{Flow} Kotlin. La natura dei \texttt{Flow} è di essere strutture dette \textit{cold}, ovvero che l'invio vero e proprio di valori avviene solamente
nel momento in cui esso inizia ad essere collezionato. Una volta che esso viene collezionato, viene anche consumato, risultando in un impossibilità di condivisione dello stesso.
Nel caso di un server GraphQL, il quale necessita di rispondere ad un numero esiguo di client con lo stesso tipo di informazioni, è necessaria una struttura che sia condivisibile
cioè che non venga consumata nel momento in cui viene collezionata. All'interno di questo contesto si inseriscono i \texttt{MutableSharedFlow}, i quali estendono le funzionalità
di un \texttt{Flow}; in particolare essi forniscono un supporto di condivisione del flusso di dati, senza che venga consumato dai collettori, ma essendo una struttura mutabile,
l'invio è detto \textit{hot}, cioè che il flusso di dati viene spedito nonostante ci sia qualche collettore dall'altra parte. Per quanto questa caratteristica sia in alcuni casi
non voluta, poiché il server invierà sempre dati nel flusso anche se nessun client ha effettuato la sottoscrizione, essa risulta utile per tutti quei casi dove si può manifestare
\textit{backpressure}, ovvero che il client non riesce a stare al passo con i dati inviati dal server. In questo contesto infatti, siccome gli elementi nel flusso vengono inviati
sempre e in modo costante, il server non verrà bloccato da client che richiedono molto tempo per l'elaborazione dei dati stessi, lasciando il compito di gestire il \textit{backpressure}
lato client, come descritto in \Cref{ssec:backpressure}.

\paragraph{Client}


\section{Verifica}\label{sec:testing}

