%!TEX root = ../thesis-main.tex
\chapter{Analisi}\label{cap:analysis}
\section{Requisiti}\label{sec:requirements}
L'applicativo si pone come obiettivo la realizzazione di un infrastruttura server all'interno della \ac{JVM}, in grado
di esporre verso l'esterno funzionalità e dati riguardanti simulazioni eseguite attraverso il simulatore Alchemist.
Con infrastruttura server si intende una componente software in grado di ricevere, elaborare e rispondere attivamente a richieste da parte
di client esterni al sistema, o quantomeno in un livello d'astrazione superiore al simulatore stesso (e.g. un web renderer che sfrutti il
server costruito per visualizzare all'interno di un browser dati riguardanti la simulazione).

\paragraph{Requisiti Funzionali} Innanzitutto, il suddetto server si dovr\aacc{a} occupare di esporre un interfaccia comune per l'instaurazione
di comunicazioni. Una volta connessi, agli agenti esterni \aacc{e} necessario specificare quali tipi di operazioni sono permesse, e in quali
modalit\aacc{a} le risposte verranno consegnate. Principalmente devono essere esposte tre diversi tipi di operazioni:
\begin{itemize}
    \item Recupero dati: operazioni cosiddette \textit{one-shot} per richiedere dati riguardo la simulazione corrente.
    \item Recupero dati persistente: operazioni di recupero continuativo nel tempo di dati riguardo la simulazione. I dati verranno spediti
        al richiedente a fronte di cambiamenti di elementi d'interesse nel dominio applicativo.
    \item Modifica: operazioni con lo scopo di effettuare modifiche al modello applicativo, in grado di provocare \textit{side effect} all'interno
        della simulazione corrente.
\end{itemize}
%
La comunicazione quindi terminerà con il verificarsi di uno dei seguenti eventi:
\begin{itemize}
    \item L'operazione di recupero dati o di modifica ha avuto successo o fallisce.
    \item L'operazione di recupero dati persistente raggiunge una condizione di terminazione (e.g. l'oggetto osservato viene eliminato).
    \item La simulazione termina.
\end{itemize}
%
Essendo il simulatore Alchemist in grado di eseguire più simulazioni simultaneamente (\textit{simulation batch}), il server deve essere in grado
di rispondere a più richieste nello stesso istante, potenzialmente anche da client distinti. Inoltre, per via della natura variable delle diverse
simulazioni, le operazioni esposte devono essere in grado di adattarsi dinamicamente in base al tipo dell'incarnazione.

\paragraph{Requisiti non Funzionali}
\begin{itemize}
    \item Dovendo condividere le risorse con il simulatore stesso, il server non deve costituire un eventuale collo di bottiglia che abbia
        un impatto negativo sulle prestazioni del sistema complessivo.
    \item L'applicativo sviluppato deve essere compatibile con un ambiente multi piattaforma.
    \item Sarebbe auspicabile che il server implementi meccanismi di estensione per agevolare l'integrazione con possibili cambiamenti del dominio applicativo.
\end{itemize}

\section{Analisi dei requisiti}
I requisiti illustrati nella sezione precedente, mettono in luce alcune caratteristiche interessanti che il sistema deve soddisfare. Primo su tutti,
il vincolo sull'utilizzo della \ac{JVM} per lo sviluppo di un server, il quale necessiterà di meccanismi in grado di instaurare connessioni
con diversi attori e sia in grado di rispondere adeguatamente ad ogni richiesta. Queste richieste dovranno successivamente essere elaborate e provocare
una risposta, sia essa di successo o fallimento. Per la gestione di questo tipo di richieste è necessario quindi valutare le alternative architetturali
definite negli anni che siano al contempo compatibili con i vincoli implementativi.

\subsection{Analisi delle Architetture API}\label{ssec:api-analysis}
Nel corso degli anni sono state proposte diverse alternative di pattern architetturali in grado di guidare gli sviluppatori a costruire un software
distribuito in rete in grado di sopperire al numero sempre maggiore di richieste e risorse.
In questa analisi verranno presi in considerazione i due principi architetturali più utilizzati: \textbf{REST} e \textbf{GraphQL}.

\paragraph{REST}\label{par:rest}
Intorno alla fine degli anni novanta e inizio anni duemila, con la diffusione sempre maggiore di internet e di siti per uso generale, nasce l'esigenza
di definire uno standard architetturale di software basati sulla rete: \ac{REST}~\cite{rest}. \ac{REST} definisce un insieme di principi architetturali
mirati alla semplicità e alla scalabilità del software. Al centro di questi principi c'è l'idea di considerare tutto come una \textit{risorsa}, a cui si
accede attraverso URL standardizzati ed univoci.
Le API \ac{REST} operano attraverso l'utilizzo dei metodi standard HTTP per eseguire le cosiddette operazioni CRUD (\textit{Create, Read, Update, Delete}) sulle risorse; in particolare utilizza:
\begin{itemize}
    \item \textbf{GET} per leggere o recuperare informazioni;
    \item \textbf{POST} per creare nuove risorse;
    \item \textbf{PUT} o \textbf{PATCH} per aggiornare risorse presenti;
    \item \textbf{DELETE} per eliminare risorse.
\end{itemize}

Di norma, si dice che \ac{REST} è ``\textit{Stateless}'', ovvero che ogni richiesta da client a server deve contenere tutte le informazioni necessarie
per comprenderla e processarla. Questo significa che il server non conserva alcuna informazione sullo stato del client, e che ogni richiesta da parte
del client è da considerarsi come una nuova richiesta indipendentemente dalle precedenti effettuate.

\paragraph{GraphQL}\label{par:gql}
GraphQL\footnote{http://facebook.github.io/graphql/} viene sviluppato a partire dal 2012 da Facebook, e si pone come obiettivo la definizione di specifiche e principi architetturali
per affrontare sfide incontrate con le tradizionali architetture software basate su \ac{REST}. La principale motivazione dietro GraphQL nasce dal
contesto del social network Facebook, dove molto spesso la larghezza di banda e risorse computazionali sono molto limitate. GraphQL quindi delinea
linee guida per la realizzazione di architetture che permettano al client la flessibilità di poter richiedere esattamente i dati di cui necessitano.
Questo avviene mediante l'impiego di un linguaggio ad hoc per la definizione di \textit{query}, le quali specificano esattamente quali dati recuperare
dal singolo \textit{endpoint} delle API esposte dal server: in GraphQL infatti non si deve specificare un URL per ogni risorsa, ma queste vengono 
recuperate a partire da un singolo URL univoco (e.g. \texttt{/graphql}) in combinazione con una query GraphQL.
Le operazioni all'interno di GraphQL sono:
\begin{itemize}
    \item \textbf{Query} per leggere determinati dati;
    \item \textbf{Mutation} per modificare uno o più dati;
    \item \textbf{Subscription} per sottoscriversi ai cambiamenti di un dato, attraverso aggiornamenti in tempo reale (discussa in \Cref{ssec:real-time}).
\end{itemize}
Non esistendo un vincolo ad una specifica base di dati, per poter operare sui dati GraphQL necessita di lavorare su uno \textit{schema}, definito
attraverso uno \ac{SDL} che descrive la struttura dei dati attraverso i loro tipi e campi all'interno dello schema. Grazie a ciò, ogni operazione
viene dapprima validata sullo schema, e successivamente eseguita sul server fornendo così un meccanismo di \textit{type safety} in grado di bloccare
in anticipo richieste GraphQL malformate.

\paragraph{Conclusioni}\label{par:rest-vs-gql}
Oggi queste due architetture convivono all'interno di molti sistemi, poiché ognuna offre i relativi pro e contro. Primo su tutti, \ac{REST} risulta più semplice
da un punto di vista implementativo, perché non richiede la definizione e manutenzione di uno schema che descriva i dati. Allo stesso tempo grazie a
quest'ultimo GraphQL è in grado di fornire meccanismi che prevengano il fenomeno di \textit{over-fetching}, del quale soffrono invece sistemi RESTful.
Per quanto riguarda l'efficienza effettiva di questi sistemi, non è possibile determinare a priori uno sull'altro, poiché per quanto GraphQL riesca
a privilegiare scenari dove la scarsa banda è notevole (e.g. applicazioni per dispositivi mobili, dispositivi IoT), in alcuni sistemi risulta
più efficiente un'architettura che possa sfruttare al meglio il \textit{caching} dei dati, tipico dei sistemi \ac{REST}~\cite{gql_vs_rest}.

In sintesi, l'adozione di una delle due architetture dipende fortemente dalla natura dell'applicativo, rendendo necessaria la conoscenza
dei pro e contro di un sistema maturo e largamente diffuso come REST, oppure un paradigma relativamente nuovo e ancora in via di sviluppo,
ma dalle grandi potenzialità.

\subsection{Dati in Tempo Reale}\label{ssec:real-time}
Il sistema deve fornire le informazioni riguardo le simulazioni sperabilmente in modo asincrono ed efficiente, sia lato server sia lato client.
In sistemi software dove avviene una costante richiesta da parte di un client di dati in tempo reale, pu\aacc{o} essere effettuata nei casi pi\aacc{u} semplici
attraverso ci\aacc{o} che viene definito ``\textit{long polling}''. In questo scenario, come raffigurato in \Cref{fig:server-polling}, viene instaurata
una comunicazione per ogni volta che il client necessita di un dato, potenzialmente provocandone il blocco in attesa della risposta. \aacc{E} da subito
evidente come se le comunicazioni sono frequenti, l'\textit{overhead} causato dalla instaurazione della comunicazione e attesa di ricezione dei dati
possono risultare in un sistema ben poco efficiente.

%----------------------------------------%
\centerimage{figures/server-long-polling.png}{Richiesta di dati in tempo reale attraverso \textit{long polling}.}{server-polling}{0.3}
%----------------------------------------%

\par
Un sistema pi\aacc{u} efficiente \aacc{e} costituito dal modello \textit{publish-subscribe}~\cite{pubsub}. Mediante l'utilizzo di questo
pattern, gli utenti interessati ad un certo dato si ``sottoscrivono'' ad un determinato \textit{topic}. Una volta che viene registrato un cambiamento
o viene attivato un evento riguardante quel \textit{topic}, il server ha il compito di ``pubblicare'' le informazioni aggiornate: a questo punto
tutti i client che si sono iscritti per quel determinato \textit{topic} riceveranno il dato aggiornato, senza bisogno di provocare ulteriori connessioni
con il server stesso. In questo modo viene creata una connessione persistente tra client e il buffer di spedizione, sollevando il server
dall'incarico di informare tutti i possibili client.

Per l'instaurazione di queste comunicazioni persistenti e allo stesso tempo bidirezionali, è spesso utilizzato il protocollo \textit{WebSocket}~\cite{websocket}.
Nato come soluzione per superare le limitazioni delle connessioni HTTP tradizionali, WebSocket opera su una singola connessione aperta, riducendo la
latenza e garantendo trasmissioni in tempo reale.
Questo protocollo e il pattern \textit{Publish-Subscribe} aiutano a delineare una possibile configurazione di come possa essere strutturata
l'architettura di messaggistica in tempo reale tra un client e il server Alchemist.
In \Cref{fig:server-sub} \aacc{e} mostrato qual \aacc{e} il comportamento desiderabile per
il server Alchemist, il quale dovrebbe essere in grado di rispondere in maniera asincrona a tutti i possibili client che si sottoscrivono
per ottenere dati in tempo reale riguardo gli elementi del dominio.

%----------------------------------------%
\centerimage{figures/server-subscription.png}{Richiesta dati in tempo reale sfruttando connessioni persistenti.}{server-sub}{0.4}
%----------------------------------------%

\subsection{Multiplatform}\label{sec:multiplatform}
Uno dei principali benefici del linguaggio Kotlin è il supporto alla programmazione di applicazioni multi piattaforma\footnote{https://kotlinlang.org/docs/multiplatform.html}
(e.g. Android, iOS, web).
Questa tecnologia permette di ridurre drasticamente lo sviluppo e il mantenimento dello stesso codice per differenti piattaforme, mantenendo la
flessibilità e i benefici dello sviluppo nativo con Kotlin.

Un progetto multiplatform si compone di una serie di moduli, detti \textit{source sets}, ognuno dei quali contiene al suo interno le componenti che
hanno come target una specifica piattaforma, come illustrato in \Cref{fig:kt-multiplatform}. Tutti i moduli condividono il \textit{source set} \textit{Common Kotlin}, il quale mantiene le librerie
fondamentali per il linguaggio Kotlin e tutto il codice portabile nelle diverse piattaforme.

%----------------------------------------%
\centerimagesource{figures/kmp-compilations.png}{Struttura di un progetto Kotlin Multiplatform per due piattaforme: \textit{Kotlin/JVM} e \textit{Kotlin/JS}}{kt-multiplatform}{0.8}{https://kotlinlang.org/docs/multiplatform-discover-project.html}
%----------------------------------------%

Kotlin mutliplatform può essere impiegato in molte applicazioni, ma prevalentemente si ricordano:
\begin{itemize}
    \item \textbf{Applicazioni Android, iOS e web}: in questo contesto, la logica dell'applicativo si sviluppa all'interno del modulo comune a tutte
        le piattaforme, mentre per esempio le specifiche implementazioni del front-end vengono delegate al relativo modulo. In questo modo, il codice
        specifico per ogni piattaforma viene ridotto notevolmente.
    \item \textbf{Applicazioni Web Full-stack}: in questo caso i moduli previsti possono consistere, oltre al modulo comune, un modulo \textit{Kotlin/JVM}
        per una componente server e un modulo \textit{Kotlin/JS} per delineare i comportamenti lato front-end. All'interno del modulo comune possono essere condivisi elementi del modello e
        logiche che possono essere condivisibili tra client e server.
    \item \textbf{Librerie Multiplatfom}: creando librerie con codice comune e implementazioni specifiche per le piattaforme, come JVM o web, è possibile
        utilizzarle come dipendenze in altri progetti multi piattaforma.
\end{itemize}

\section{Analisi e Modello del Dominio}\label{sec:domain-model}
L'architettura software proposta dovrà operare al di sopra del simulatore Alchemist, il quale modello è stato ampiamente esplorato all'interno di \Cref{ssec:alchemist}.
In questo contesto, è quindi dapprima necessario fornire un livello comunicativo tra il modello del dominio di Alchemist,
e le componenti del sistema server, agendo quindi come \textit{controller} all'interno del paradigma architetturale \ac{MVC}.
Il server sarà quindi responsabile dell'accesso e recupero dati nel contesto
di una simulazione, e allo stesso tempo fornire un punto di accesso per le comunicazioni con i client. Queste comunicazioni devono permette il transito sia di informazioni, sia di comandi specifici che il
server deve far attuare all'interno di  una simulazione. D'altra parte, il client deve essere in grado di stabilire e inviare correttamente questo tipo di operazioni al server.

%----------------------------------------%
\centerimage{figures/component-diagram-1.png}{Diagramma UML dei componenti dell'architettura di alto livello}{component-1}{0.8}
%----------------------------------------%
%
In figura \Cref{fig:component-1} è illustrato il diagramma UML delle componenti del modello dell'applicativo. La componente server è composta a sua volta da un livello comunicativo con ciò che rappresenta
una simulazione Alchemist, e l'esposizione di un interfaccia comune su cui i client possono stabilire una connessione. All'interno di questa interfaccia vengono stabilite tutte le operazioni eseguibili
all'interno di una simulazione. Il client comprende un modulo principale incaricato di interporsi tra le applicazioni sviluppate potenzialmente in ambienti multi piattaforma, e il punto di accesso
fornito dal server (\textit{endpoint}). Ciò che l'applicativo client vero e proprio esegue, è contenuto all'interno della componente \texttt{Client}.


Siccome Alchemist prevede diverse modalità operative, come per esempio all'interno di una GUI o all'interno di un browser, la componente server deve poter essere in grado di essere lanciata solo attraverso
specifica configurazione all'interno della simulazione. In questo modo il modulo verrà eseguito solo se richiesto esplicitamente.
