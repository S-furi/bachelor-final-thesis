%!TEX root = ../thesis-main.tex
\chapter{Conclusioni}\label{chap:conclusions}
L'insieme di quanto discusso all'interno di questo elaborato hanno portato alla costruzione di un sistema software in grado di fornire meccanismi efficienti per l'accesso
e il controllo di una simulazione all'interno di Alchemist. Questo è stato reso possibile grazie ad un attento studio delle architetture per lo sviluppo e interrogazione
di \ac{API}, prediligendo un approccio fortemente strutturato grazie a GraphQL. Il suo impiego ha reso necessaria un'analisi approfondita del dominio applicativo di Alchemist,
grazie al quale sono state definite linee guida per la loro rappresentazione all'interno di uno schema conforme alle specifiche GraphQL.

Allo stato attuale, il progetto mira a fornire una solida componente server in grado di effettuare operazioni basilari all'interno di una simulazione, la quale mira a facilitare
l'aggiunta di ulteriori operazioni e funzionalità, riducendo al minimo le modifiche all'architettura esistente. Alcune di queste funzionalità aggiuntive sono descritte all'interno
della \Cref{sec:future-works}.

Come dimostrazione del funzionamento dell'architettura software prodotta, viene presentata una simulazione presa dal tutorial di Alchemist\footnote{https://alchemistsimulator.github.io/tutorials/basics/index.html},
al cui interno è presenta una griglia di nodi in grado di giocare a \textit{dodgeball}. Il funzionamento è piuttosto semplice: il programma inizia la simulazione con una
palla, e l'obiettivo è quello di lanciarla ad un vicino in modo casuale. Ogni nodo che viene colpito, acquisisce un punto, incrementa il proprio punteggio e lancia nuovamente la palla.

\centerimage{./figures/ClientApp-Demo.png}{Applicativo client per la visualizzazione di statistiche riguardo la simulazione \textit{Dodgeball}}{client-demo}{0.7}
L'applicativo mostrato in \Cref{fig:client-demo} mostra l'interfaccia all'interno del web browser contente un pannello di controllo contenente 4 bottoni e uno spazio adibito
ad un Grafico che mostri delle caratteristiche della simulazione corrente. Il grafico è generato attraverso l'uso di Lets-Plot Kotlin\footnote{https://github.com/JetBrains/lets-plot-kotlin},
la quale è stata approfondita estensivamente all'interno dell'attività di tirocinio curricolare\footnote{https://s-furi.github.io/uni-internship}. Di seguito una breve descrizione delle
funzionalità proposte: al caricamento della pagina, è possibile sottoscriversi in un qualsiasi momento tramite il pulsante \textit{subscribe}, oppure avviare la simulazione mediante \textit{play}.
Una volta che la simulazione sarà fatta partire, all'interno del grafico verranno visualizzate le posizioni di un vicinato, in questo caso del vicinato al quale appartiene il nodo
con \texttt{id} pari a 1. La simulazione non prevede cambiamenti di posizioni durante l'esecuzione, ma muovendo il cursore al di sopra di un nodo è possibile monitorare lo stato
della molecola e concentrazione ad esso associate, dove nella prima è contenuto il numero di \textit{hit} effettuati fino a quel momento. Infine, i restanti due pulsanti hanno rispettivamente
la funzione di mettere in pausa e terminare definitivamente la simulazione. In caso di errori durante la simulazione, come per esempio la richiesta di avviamento della simulazione
dopo che è stata fatta terminare, verrano visualizzati a schermo all'utente.

\section{Lavori Futuri}\label{sec:future-works}
L'obiettivo di questo elaborato è stato quello di dimostrare la fattibilità ed implementare un'architettura software per l'esposizione di \ac{API} al di sopra del simulatore Alchemist,
provvedendo a garantire meccanismi di estensibilità sia da un punto di vista delle operazioni eseguibili, sia per quanto riguarda il modello di Alchemist.
Attraverso questo sistema software così costruito, è possibile operare su di esso per effettuare miglioramenti ed estensioni delle funzionalità, riassunte nei seguenti punti:

\begin{itemize}
    \item \textbf{N+1 \textit{problem}}: nei sistemi GraphQL si può verificare durante l'esecuzione di una \textit{query} che questa richieda una risorsa, e per ogni record
        restituito, viene effettuata un'ulteriore query per recuperare informazioni correlate, portando a un numero esponenziale di chiamate al sottosistema per il recupero dei dati,
        provocando per una operazione, ulteriori $N$ query, causando ciò che viene appunto delineato $N+1$ \textit{problem}. All'interno di GraphQL Kotlin è possibile mitigare questa
        problematica mediante appositi \textit{data loader}, i quali permettono di effettuare \textit{batching di operazioni} e offrire inoltre un livello di \textit{caching} per le
        operazioni più frequenti. Questo può quindi comportare un ampio beneficio in termini di prestazioni generali del sottosistema, sia in termini di latenza di risposte,
        sia per quanto riguarda l'\textit{overhead} generale nella risoluzione delle operazioni.
    \item \textbf{Gestione del \textit{backpressure}}: nei casi in cui il tasso di invio delle informazioni dal server al client sia maggiore di quanto il client stesso riesca a gestire,
        è necessario fornire quest'ultimo di strategie in grado di gestire il fenomeno del \textit{backpressure}. Questo può avvenire mediante politiche ad-hoc al momento della ricezione
        del \texttt{Flow}, come per esempio la scelta dell'elemento più recente all'interno del canale comunicativo, oppure prenderne solo alcuni campioni. La strategia adottata
        dipenderà da sistema a sistema.
    \item \textbf{Miglioramento strategie di Serializzazione}: per come è stato implementata la logica di serializzazione di una concentrazione, essa funziona con i tipi primitivi di Kotlin
        e le classi annotate attraverso l'annotazione \kotlin{@Serializable}. Per quanto questa soluzione sia efficace nei casi più semplici, esiste un certo numero di situazioni dove
        questo approccio risulti non viabile per via di alcuni fattori, come per esempio la presenza di ereditarietà tra gli elementi (necessitando di un \texttt{PolymorphicSerializer}) o
        anche l'impiego di istanze specifiche delle \textit{collection} di Kotlin (e.g. \texttt{ArrayList} non è serializzabile direttamente ma \texttt{List} lo è). All'interno di questo contesto
        è necessaria la valutazione di strategie più avanzate di serializzazione, le quali non hanno trovato il tempo di essere approfondite durante l'elaborazione di questo progetto.
    \item \textbf{Ampliamento del modello}: allo stato attuale, l'architettura GraphQL implementata comprende una semplice sottoparte dell'intero modello di Alchemist. L'aggiunta di ulteriori
        elementi come \textit{linking rule}, \textit{action} e \textit{reaction} renderanno il sistema in grado di rappresentare l'intero meta-modello di Alchemist con la corretta interpretazione
        in termini di specifiche GraphQL.
\end{itemize}
