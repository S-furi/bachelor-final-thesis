%!TEX root = ../thesis-main.tex
\begin{abstract}
    All’interno dei sistemi software moderni è di cruciale importanza la
    possibilità di monitorare e controllare un’intero sistema nella maniera più
    efficiente possibile. I principali meccanismi di controllo e interrogazione
    dei sistemi possono comprendere il design e l’implementazione di \ac{API},
    le quali costituiscono un solido punto di appoggio per le comunicazioni tra
    le diverse componenti sotware di un sistema. All’interno di questo
    elaborato, viene presentata la realizzazione di un servizio di API
    all’interno di Alchemist, un simulatore stocastico realizzato all’interno
    della \ac{JVM}.

    Il servizio di API mira ad esporre verso l’esterno un insieme di
    informazioni inerenti una simulazione all’interno di Alchemist, e allo
    stesso tempo fornisce meccanismi di controllo della stessa, mantenendo un
    implementazione che non sia dipendente dalla piattaforma utilizzata
    mediante l’uso di Kotlin Mutliplatform.

    All’interno di questo documento, verranno illustrate le strategie di design
    e architetturali per la realizzazione di un tale sistema attraverso il
    paradigma GraphQL, il quale fornisce un \textit{Query Language} in grado di
    stabilire con esattezza la struttura dei dati che un client deve ricevere
    dal sistema di API. Sono quindi approfondite motivazioni dietro la scelta
    di tale paradigma, e vengono  definite le operazioni di massima che il
    sistema deve fornire. Successivamente verrano coperte estensivamente le
    sfide implementative e di compatiblità causate dalla natura complessa di
    Alchemist, e la rigida e semplice struttura di uno schema GraphQL. Infine,
    sono illustare le future estensioni possibili del sistema costruito,
    illustrandone scopi e benefici che possono apportare al sistema,
    richiedendo pochi o nessun cambiamento all’architettura proposta.
\end{abstract}
